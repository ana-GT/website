% LaTeX resume using res.cls
\documentclass[margin]{res}
%\usepackage{helvetica} % uses helvetica postscript font (download helvetica.sty)
\usepackage{times}   % uses new century schoolbook postscript font 
\setlength{\textwidth}{5.1in} % set width of text portion

\begin{document}

% Center the name over the entire width of resume:
 \moveleft.5\hoffset\centerline{\sl {\huge\bf Ana C. Huam\'an Quispe} }
 \smallskip

% address begins here
% Again, the address lines must be centered over entire width of resume:
\bigskip 
\bigskip 

\address{ {\it Humanoid Robotics Lab} \\ Center for Robotics and Intelligent Machines (RIM)\\ Georgia Institute of Technology\\
        www.cc.gatech.edu/$\sim$ahuaman3/ \\ahuaman3@gatech.edu}
\address{{\it Current Address} \\ 1033 Tumlin St, Apt C3 \\ GA 30318, Atlanta - US \\
         (404) 202-4843 }

% Draw a horizontal line the whole width of resume:
 \moveleft\hoffset\vbox{\hrule width\resumewidth height 1pt}\smallskip

\begin{resume}
 
\section{RESEARCH INTERESTS} My Research focuses on enabling humanoid robots to perform useful tasks.
I am particularly interested in two interrelated areas: Developing algorithms for robot manipulation in everyday environments, and robot locomotion for humanoids.

\section{EDUCATION} 

				{\sl \textbf{Georgia Institute of Technology - USA} \hfill August 2010 - present} \\
                PhD in Robotics 
                       
				{\sl \textbf{Georgia Institute of Technology - USA} \hfill August 2012 - May 2013} \\
                Master in Computer Science \\
                (with specialization in Computational Perception and Robotics) 
                
				{\sl \textbf{Universidad Nacional de Ingenieria - Per\'{u}} \hfill August 2003 - August 2008} \\
                Bachelor of Science in Mechatronics Engineering  \\
     
\section{RESEARCH EXPERIENCE} 

                {\sl Graduate Research Assistant} \hfill            Fall 2010 - Present \\
                {\sl Humanoids Lab - Georgia Institute of Technology}\\ 
				Currently I am working in generating smooth locomotion gaits to be applied in a real humanoid robot (Hubo). Previous related work included planning algorithms for manipulation using redundant robotic arms.

                {\sl Research Programmer} \hfill            Summer 2009 \\
                {\sl BioRobotics Lab - Carnegie Mellon University}

                {\sl Programmer in Projects for NAO Robots} \hfill        Fall 2009 \\
                CMRobobits Course - Carnegie Mellon University

\section{WORK\\ EXPERIENCE}
               {\sl OpenCV - Google Summer of Code \hfill  Summer 2011\\
                \textit{Position:} Student Developer }\\
				I wrote tutorials and source code using OpenCV (C++ Computer Vision Library)
				My work is currently hosted in the official documentation website of the library (tutorials section)

\section{TEACHING\\ EXPERIENCE}

                \begin{itemize}  \itemsep -2pt %reduce space between items
                \item{\sl Volunteer - Counselor}\hfill  Fall 2011 - Fall 2012\\
                \textit{Institution:} Whiz Kidz Science and Technology Centers \\
                \textit{Course:} STEM classes for K-9 kids 
                
                \item{\sl Tutor}\hfill  Fall 2011 - Fall 2012\\
                \textit{Institution:} GT-SHPE (Society of Hispanic Professionals Engineers) \\
                \textit{Course:} Tutoring for middle and high school kids in Atlanta 
                
                \item{\sl Teaching Assistant}\hfill  Spring 2011\\
                \textit{Institution:} Georgia Institute of Technology \\
                \textit{Course:} CS 3630 -Intro to Perception and Robotics (Undergraduate) 
                                 
                 \end{itemize}

\section{PUBLICATIONS}

                \sl Deterministic Motion Planning for Redundant Robots along End-Effector Paths \\
                \textit{Ana Huam\'{a}n Quispe and Mike Stilman} \\
                \textit{2012 IEEE-RAS International Conference on Humanoid Robots} 
				
\section{TECHNICAL REPORT}
				\sl  Diverse Workspace Path Planning for Robot Manipulators \\ 
                \textit{Ana Huam\'{a}n Quispe and Mike Stilman} \\
                \textit{URL: http://smartech.gatech.edu/xmlui/handle/1853/44264}\\
	            \textit{July 2012}  

\section{COMPUTER \\ SKILLS} 
                 \begin{itemize}  \itemsep -2pt %reduce space between items
                 \item{\sl Operating Systems:} Unix (preferred) and Windows. 
                 \item{\sl Computer Languages:} C++ and C. Exposure to Lisp, Java and Python.
                 \item{\sl Robot Programming:} ROS and Gazebo for dynamic simulation.
                 \item{\sl Libraries: OpenCV, Point Cloud Library (PCL), Eigen}
				\item{\sl SCM: git and svn}
				\item{\sl Scientific Applications:} MATLAB/Simulink, Octave, Maple.
                 \item{\sl Microcontrollers Programming:} MPLAB and PICC.
                 \item{\sl Electronic Design:} Proteus (ISIS and ARES), OrCAD, Circuit Maker.
                 \item{\sl Technical Drawing:} AutoCAD, SolidWorks.
                 \item{\sl Internet Development:} HTML. 
                 \end{itemize}


\section{RELEVANT \\ COURSEWORK} 
                 \begin{itemize}  \itemsep -2pt %reduce space between items
	             \item{Computability and Algorithms} \hfill GaTech Spring 2013 
	             \item{Humanoid Robotics} \hfill GaTech Spring 2013 	             
                 \item{\sl 3D Reconstruction and Mapping} \hfill GaTech Fall 2012 
                 \item{\sl Implementation and Control of Robotic Systems} \hfill GaTech Fall 2012 
                 \item{\sl Linear Control} \hfill GaTech Fall 2011                  
                 \item{\sl Computer Vision} \hfill GaTech Fall 2011  
                 \item{\sl Artificial Intelligence} \hfill GaTech Spring 2011                                                    
                 \item{\sl Robot Intelligence: Planning in Action} \hfill GaTech Fall 2010                  
                 \end{itemize}

\section{LANGUAGE \\ SKILLS} 
                 \begin{itemize}  \itemsep -2pt %reduce space between items
                 \item{{\sl Spanish:}} Native 
                 \item{\sl English:} Fluent
                 \item{\sl German:} Basic 
                 \end{itemize}




\end{resume}

\end{document}




